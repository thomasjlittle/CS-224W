\documentclass{article}

\usepackage[final]{neurips_2019}

\usepackage[utf8]{inputenc}
\usepackage[T1]{fontenc}
\usepackage{hyperref}
\usepackage{url}
\usepackage{booktabs}
\usepackage{amsfonts}
\usepackage{nicefrac}
\usepackage{microtype}
\usepackage{graphicx}
\usepackage{xcolor}
\usepackage{lipsum}

\newcommand{\note}[1]{\textcolor{blue}{{#1}}}

\title{
  Predicting Block Copolymer Properties Using Graph Neural Networks \\
  %\vspace{1em}
  %\small{Project Category:} \\
}

\author{
  Name: \\
  SUNet ID: 06510735\\
  Department of Civil and Environmental Engineering \\
  Stanford University \\
  \texttt{tjlittle@stanford.edu} \\
  % Examples of more authors
  \And
  Name: \\
  SUNet ID: \\
  Department of Computer Science \\
  Stanford University \\
  \texttt{name@stanford.edu} \\
}

\begin{document}

\maketitle

% \begin{abstract}
%   Required for final report
% \end{abstract}


\note{This template is built on the NeurIPS 2019 template\footnote{\url{https://www.overleaf.com/latex/templates/neurips-2019/tprktwxmqmgk}} and provided for your convenience. Your proposal should include a 300-500 word description of what you plan to do. Presenting pointers to one relevant dataset and one example of prior research on the topic are a valuable (optional) addition.}

\section{Key Information to include}

\begin{itemize}
    \item External collaborators or mentors (if you have any):
    \item Sharing project with another class:
\end{itemize}

\section{Motivation}
Provide a clear motivation and build up to your problem statement. What problem are you tackling? Is this an application or a theoretical result?

\section{Methodology}
What machine learning techniques are you planning to apply or improve upon?

\section{Intended experiments}
What experiments are you planning to run? How do you plan to evaluate your machine learning algorithm? Here is how to use a reference: \cite{attention_vaswani_17} if you want to include them. Include the references in the references.bib file.

\bibliographystyle{unsrt}
\bibliography{references}

\end{document}
